\documentclass{beamer}
\usetheme{Boadilla}
\setbeamertemplate{footline}[frame number]{}
\setbeamertemplate{navigation symbols}{}

\title[MAC-MIGS Induction: Collaboration]{Version Control: Sharing and working together  on \LaTeX + code}
%\date{29th September, 2020}
\author[L. Banjai]{ Lehel Banjai (\texttt{l.banjai@hw.ac.uk})}
%silly local change
\usepackage{ulem}

\begin{document}

\frame{\titlepage
}

\begin{frame}{Version Control/Data Share}
  Why do we need it?

  \begin{itemize}
  \item More people working on the same project in different locations  and at the same time.
   \item Easy access to past versions.
\item Resolving conflicting changes.
    \item Share code and data with wider community.
    \item Funding agencies and journals require more and more the reproducibility of results: This requires access to both data and code.
  \end{itemize}\pause

  \begin{block}{}
Much of the version control software comes (no surprise) from the coding community: large codes, many programmers, many versions, integration of different versions etc.   
  \end{block}
\end{frame}

\begin{frame}{Version control/data share software}

 \begin{itemize}
 \item Email back-and-forth and attach date to filename, e.g., {\texttt{currentwork27092023.tex}} 
   \begin{itemize}
   \item You may laugh, but probably most common.
   \item OK if only a few versions of a few files likely.
     \item Past versions in your Inbox
      \item Not good for sharing code.\pause
   \end{itemize}
\item Dropbox 
  \begin{itemize}
  \item Basic functionality, (possible?) security issues.
  \item Limited free storage.
   \item Constantly updates the repository - both positive and negative.
  \item Extremely easy to  use  with one or two collaborators.  
  \item Not really intended for wide dissemination of code.\pause
  \end{itemize}
\item Subversion (SVN)
  \begin{itemize}
  \item The real deal -- all version control power you may wish.
  \item Basic functionality easy to use (I've only used command line in Linux).
  \item Open source (Apache License)
  \item Requires a host server.
  \end{itemize}
 \end{itemize}
\end{frame}

\begin{frame}{Version control software/data share ctd.}
  \begin{itemize}
  \item Microsoft Teams (not really version control)
    \begin{itemize}
    \item Simply data storage in the cloud (many other options there).
    \item Good for simpler, not very dynamic projects with few collaborators.
    \item Have video conferences.
   \item Easy to use.
%    \item \sout{I've found it useful for interaction with PhD and MSc students.}      \pause
    \end{itemize}
\item Git/GitHub
  \begin{itemize}
  \item Git as good as SVN for version control.
  \item Git: free and open source (GNU General Public Version 2)
   \item The most used version control.   
   \item Ideal for sharing code.
   \item Integrated support in many IDEs (e.g., Pycharm)
   \item GitHub a distributive repository. (No need for host server)
    \item Repository on GitHub can be both private and public.
    \item Somewhat trickier to begin with, but once all set up, the basic functionality easy to use.
    \item GitHub recently bought by Microsoft \dots
    \item Alternatives to GitHub: GitLab, SourceForge, BitBucket \dots
  \end{itemize}
  \end{itemize}
\end{frame}

\begin{frame}{Online collaborative \LaTeX editors}
 Can type at the same time in the document and see real time the changes by the collaborators.

  \begin{itemize}
  \item Overleaf
    \begin{itemize}
    \item Easy to use.
     \item Slow compilation.
    \item With free version limit on number of collaborators.
      \item Both HW and UoE have the paid version.
      \item As with many things now, it has Git/Github integration\pause
    \end{itemize}
\item Papeeria (free version)
  \begin{itemize}
  \item Unlimited number of collaborators per project.
  \item Only one active private project.
\item Git sync
    \end{itemize}
  \end{itemize}
\end{frame}

\begin{frame}{Setting up a new local repository with Git}
  \begin{itemize}
  \item If not available install Git (In Debian \texttt{sudo apt get install git})
\item Create an account on GitHub (\texttt{github.com}).
\item Create directory, enter the directory and run \\
  \begin{center}
    \texttt{git init}
  \end{center}
\item Write some stuff (include a ReadMe file) and add with \\
  \begin{center}    
\texttt{git add foo} 
  \end{center}
\item Commit changes to repository\\\begin{center} \texttt{git commit -m="description of change being commited"}\end{center}
\item Now you have local version control. To allow other people to use, need  to create a GitHub repository.
  \end{itemize}
 \end{frame}

 \begin{frame}{Connecting to GitHub}
   \begin{itemize}
   \item Go to \texttt{github.com} and log in.
   \item Create new repository (top right). you can choose whether to make it private or public.
   \item Go back to your folder, and follow the instructions, e.g.,\\
     \begin{center}
       \begin{tabular}{l}
       \texttt{git remote add origin https://github.com/lehelb/Test.git}\\
       \texttt{git branch -M main}\\
       \texttt{git push -u origin main}         
       \end{tabular}
     \end{center}
   \end{itemize}
 \end{frame}

 \begin{frame}[fragile]{Basic functionality}
\begin{verbatim}
#to add new file
git add new_file
# remove file
git rm old_file
# git status
git status
# to commit all at any point added files
git commit -a
# to push to git-hub
git push
# if changed file but want to revert to the original version
git checkout -- filename
# pull if someone else changed something
git pull
#to remember token
git config credential.helper store
# to ignore files add them to .gitignore
\end{verbatim}
 \end{frame}

 \begin{frame}{Further functionality}
   \begin{itemize}
   \item You can add collaborators on GitHub. 
   \item You can clone other people's repositories if they are public. Unless you are added as collaborator you cannot push changes.
\item If you want to to work on someone else's repository {\em fork} it on GitHub.
\item etc.
\item There's a huge amount of information and tutorials online.
\item The repository we created in the talk is at \texttt{https://github.com/lehelb/MacMigsCollaborate}.
   \end{itemize}
 \end{frame}
\end{document}
